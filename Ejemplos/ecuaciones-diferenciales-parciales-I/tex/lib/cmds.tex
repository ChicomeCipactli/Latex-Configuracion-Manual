% ------- Conjuntos -------

\newcommand{\set}[1]{   % Descripción conjunto cualquiera (llaves)
    \left\{
        #1
    \right\}
}

    % Conjuntos de números

\newcommand{\N}{    % Números naturales
    \mathbb{N}
}
\newcommand{\Z}{    % Números enteros
    \mathbb{Z}
}
\newcommand{\Q}{    % Números racionales
    \mathbb{Q}
}
\newcommand{\I}{    % Números irracionales
    \mathbb{I}
}
\newcommand{\R}{    % Números reales
    \mathbb{R}
}
\newcommand{\C}{    % Números complejos
    \mathbb{C}
}
\newcommand{\F}{    % Campo arbitrario ?
    \mathbb{F}
}

    % Topología

\newcommand{\Fr}{
    \mathrm{Fr}\,
}

\newcommand{\cl}[1]{
    \overline{#1}
}
\newcommand{\interior}{
    \mathrm{int}\,
}

    % Conjuntos asociados a funciones

\newcommand{\im}[1]{    % Imagen de una función
    \mathrm{Im} \,
}
\newcommand{\dom}[1]{   % Dominio de una función
    \mathrm{Dom} \,
}

    % Conjuntos asociados a funciones de Álgebra Lineal

\newcommand{\kernel}[1]{   % Kérnel de una transformación lineal
    \mathrm{Ker} \,
}
\newcommand{\setspan}[1]{   % Espacio generado por un conjunto
    \mathrm{span} \,
}

% ------- Funciones -------

    % Normas

\providecommand{\abs}[1]{   % Valor absoluto
    \left\vert
        #1
    \right\vert
}
\providecommand{\norm}[1]{  % Norma
    \left\lVert
        #1
    \right\rVert
}
\providecommand{\prodint}[1]{   % Producto interior
    \left\langle
        #1
    \right\rangle
}

    % Probabilidad

\newcommand{\Esp}[1]{   % Esperanza matemática
    E
    \left[
        #1
    \right]
}
\newcommand{\Prob}[1]{  % Probabilidad
    P
    \left[
        #1
    \right]
}
\newcommand{\ind}[2]{   % Función indicadora / Función característica
    1_{#1}
    \left(
        #2
    \right)
}

    % dimensiones de Álgebra Lineal

\renewcommand{\dim}[2][]{   % Dimensión
    \mathrm{dim} \,
}
\newcommand{\rango}[1]{ % Rango (dimensión del espacio imagen)
    \mathrm{rango} \,
}
\newcommand{\nulidad}[1]{   % Nulidad (dimensión del kérnel)
    \mathrm{nulidad} \,
}

    % Números complejos
\newcommand{\repart}[1]{
    \mathrm{Re}
    \left(
        #1
    \right)
}
\newcommand{\impart}[1]{
    \mathrm{Im}
    \left(
        #1
    \right)
}

% ----- Mapeos -----

\newcommand{\To}{   % Flecha larga
    \longrightarrow
} 
\newcommand{\mTo}{ % Flecha larga con cola (manda a)
    \longmapsto
} 

% ----- Lógica -----

\newcommand{\ssi}{  % Si y solo si (<=>)
    \Leftrightarrow
}
\newcommand{\ent}{  % Entonces (=>)
    \Rightarrow
}
\newcommand{\Ent}{  % Entonces largo (==>)
    \Longrightarrow
}
\newcommand{\Ssi}{  % Si y solo si largo (<==>)
    \Longleftarrow
}

% ----- Letras especiales ------

\newcommand{\eps}{  % Épsilon bonita (siempre varépsilon)
    \varepsilon
}

\newcommand{\uphi}{
    \phi
}
\renewcommand{\phi}{
    \varphi
}
