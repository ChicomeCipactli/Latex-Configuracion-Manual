% Definiciones de teoremas, definiciones...

% \newtheorem{axiom}{Axioma}
% \theoremstyle{definition}
% \newtheorem{definition}{Definición}[section]
% \newtheorem{theorem}{Teorema}[section]
% \newtheorem{lemma}{Lema}[section]
% \newtheorem{corollary}[theorem]{Corolario}
% \newtheorem{proposition}[theorem]{Proposición}
% \theoremstyle{remark}
% \newtheorem*{note}{\bfseries Nota}
% 
% \crefname{definition}{Definición}{Definiciones}
% \crefname{axiom}{Axioma}{Axiomas}
% \crefname{theorem}{Teorema}{Teoremas}
% \crefname{lemma}{Lema}{Lemas}
% \crefname{corollary}{Corolario}{Corolarios}
% \crefname{proposition}{Proposición}{Proposiciones}
% \crefname{section}{Sección}{Secciones}
% 
% \newenvironment{solution}{\begin{proof}[Solución]}{\end{proof}}
% \renewcommand{\qedsymbol}{$\blacksquare$}
% 
% \newcommand{\QED}{$\hfill\blacksquare$} % Pone el cuadrito negro hasta el final de la línea
% \newcommand{\EndSol}{$\hfill\square$} % Cuadrito blanco al final al final de la línea

\renewenvironment{proof}[1][\textbf{\textit{Demostración}}]{
     \par\medskip\noindent \textcolor[rgb]{0.2,0,0.6}{#1.}
     \rmfamily
}{
    \par
    % $\hfill\textcolor[rgb]{0.2,0,0.6}{\textbf{QED}}$\\
    $\hfill\textcolor[rgb]{0.2,0,0.6}{\blacksquare}$\\
}

\newenvironment{solution}[1][\textbf{\textit{Solución}}]{
     \par\medskip\noindent \textcolor[rgb]{0.2,0,0.6}{#1.}
     \rmfamily
}{
    \par
    % $\hfill\textcolor[rgb]{0.2,0,0.6}{\textbf{S}}$\\
    $\hfill\textcolor[rgb]{0.2,0,0.6}{\square}$\\
}

\renewenvironment{boxed} % [1][]
{
    \begin{center}
    \begin{tabular}{|p{0.7\textwidth}|}
    \hline
}{ 
    \\
    \hline
    \end{tabular} 
    \end{center}
}

\newcounter{problem}
\newenvironment{problem}[1][\theproblem]{
    \begin{tcolorbox}[colback=red!50,colframe=red!50]
        \refstepcounter{problem}
        % \par\medskip
        % \noindent 
        % \color[rgb]{0.1,0.1,0.1}
        % \color[rgb]{0.1,0,0.5}
        \textbf{Problema~#1.}
        \rmfamily
}{
    \end{tcolorbox}
    \medskip
}
