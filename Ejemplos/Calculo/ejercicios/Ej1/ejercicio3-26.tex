Sea \( f: [a,b] \to \R \) integrable y no negativa, y sea \( A_f = \set{( x,y ) : 
a\leq x \leq b \text{ y } 0\leq y \leq f(x) } \). Demuestra que \( A_f \) es medible de
Jordan, y tiene área \( \integral{a}{b}{f} \).
\begin{proof}
    La frontera de $A_f$ es
    \begin{align*}
        \partial A_f 
        &=
        \set{
            (a,y)
            :
            0\leq y \leq f(a)
        }
        \cup
        \set{
            (b,y)
            :
            0\leq y \leq f(b)
        }
        \\
        &\quad
        \cup
        \set{
            (x,f(x)):
            a\leq x \leq b
        }
        \cup
        \set{(x,0):a\leq x\leq b}.
    \end{align*}
    Sea \( \eps >0 \), si $0 < \delta < \frac{\eps}{2f(a)+4} $, el rectángulo \( R= [ a - \delta , 
    a + \delta ] \times \left[ - 1, f(a) + 1 \right]   \) contiene al conjunto
    \( 
        \set{
            (a,y)
            :
            0\leq y \leq f(a)
        }
    \) y tiene volumen \( \mathrm{vol}(R)=
    2\delta ( f(a) + 2) = \delta (2f(a)+4)< \eps \). Por lo tanto
    \( 
        \set{
            (a,y)
            :
            0\leq y \leq f(a)
        }
    \) tiene contenido cero. 

    De forma similar, tomando \( \delta < \frac{\eps}{2f(b)+4} \) el conjunto
    \( 
        \set{
            (b,y)
            :
            0\leq y \leq f(b)
        }
    \), está contenido en el rectángulo \( \left[ b-\delta, b + \delta \right] \times
    \left[ -1,f(b)+1 \right]\), con volumen menor que \( \eps \). Así que este
    conjunto también 
    tiene contenido 0.

    El conjunto \( \set{(x,f(x)):a\leq x \leq b} \) es la gráfica de una función integrable, 
    por 
    cual tiene contenido 0. 
    La función constante 0 también es integrable, por lo que su gráfica 
    \( \set{(x,0):a\leq x \leq b }\) tiene contenido 0.

    Elc onjunto 
    \( \partial A_f \) es la unión finita de conjuntos de contenido 0. Por lo tanto tiene 
    contenido cero y \( A_f \) es medible de Jordan con área
    \begin{align*}
        \mathrm{area}(A_f)
        &=
        \int_{A_f}
        1
        =
        \int_{[a,b]\times [0, \max{f}]}
        \chi_{A_f}\\
        &=
        \int_{a}^b
        \int_{0}^{\max f}
        \chi_{A_f}(x,y) dy\,dx\\
        &=
        \int_{a}^b
        \left(
        \int_{0}^{f(x)}
        \chi_{A_f}(x,y) 
        dy
        +
        \int_{f(x)}^{\max f}
        \chi_{A_f}(x,y)
        dy\,
        \right)
        dx.
    \end{align*}
    Pero siempre que \( x\in [a,b] \) 
    cuando $y> f(x)$, $\chi_{A_f}(x,y)=0$, y su integral es 0, y cuando 
    \( 0 \leq y \leq f(x) \), \( \chi_{A_f}(x,y)=1 \). Por lo tanto
    \begin{align*}
        \mathrm{area}(A_f)
        &=
        \int_{a}^b
        \int_{0}^{f(x)}
        \chi_{A_f}(x,y) 
        dy
        \,
        dx\\
        &=
        \int_{a}^b
        \int_{0}^{f(x)}
        1\,
        dy
        \,
        dx\\
        &=
        \int_a^b
        x
        {\rvert_0^{f(x)}}
        dx\\
        &=
        \int_a^b
        f(x)
        \,
        dx.
    \end{align*}
\end{proof}
