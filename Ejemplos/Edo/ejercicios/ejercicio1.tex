Encuentra la solución general de la ecuación diferencial dada
\[
    2t \sen y +y^3 e^t + \left( t^2 \cos y + 3 y^2 e ^t \right) \frac{dy}{dt} = 0
\]
\begin{solution}
    Como típicamente se escribe, identificamos las funciones
    \( M(t,y)=2t \sen y +y^3 e^t \) y a \( N(t,y)=t^2\cos y +3y^2 e ^t \).
    Notemos que 
    \( \frac{\partial M}{\partial y} = 2t \cos y + 3 y^2 e^t = \frac{\partial N}{\partial t} \),
    por lo que debe existir una función \( \Phi \) tal que \( \frac{\partial \Phi}{\partial t} =
    M\) y \( \frac{\partial \Phi}{\partial y} = N \) (Según el libro, la verdad no entendí 
    esa parte de la demostración).
    Integrando con respecto a \( t \) obtenemos 
    \begin{align*}
        c=\Phi(t,y)
        =
        \int M(t,y)
        \, dt
        &=
        \int 2t \sen y +y^3 e^t
        \, dt\\
        &=
        t^2\sin y+y^3 e^t + h(y),\quad\text{para alguna función $h$.}
    \end{align*}
    % Derivado con respecto a $y$ obtenemos 
    Por otro lado, haciéndolo con respecto a \( y \) se tiene
    \begin{align*}
        c=\Phi(t,y)=
        \int N(t,y)
        \, dy
        &=
        \int 
        t^2 \cos y +3y^2 e^t
        \, dy\\
        &=
        t^2 \sin y +y^3 e^t + k(t),\quad\text{para alguna función $k$.}
    \end{align*}
    % donde \( h \) cumple que su derivada
    % \[
    %     h'(y)
    %     =
    %     N(t,y)
    %     -
    %     \int \frac{\partial M(t,y)}{\partial y} \, dt
    % \]
    Luego $h(y)=k(t)$. Derivando con respecto de $y$ vemos que $h'(y)=0$, por lo que $h$ es 
    una constante. Como \( \Phi \) es constante y la solución general de la ecuación es 
    \[
        \Phi(t,y)
        =
        t^2\sin y + y^3 e^t + C,\quad\text{para alguna $C\in\R$,}
    \]
    debe suceder que
    \[
        t^2 \sin y + y^3 e^t = C'\quad\text{ para alguna $C'\in\R$.}
    \]
\end{solution}
