Resuelve el problema de valor inicial,
\[
    3ty+y^2
    +
    \left( t^2+ty \right) \frac{dy}{dt} = 0,\quad y(2) = 1.
\]
\begin{solution}
    Expresamos como $M(t,y)=3ty+y^2$ y a \( N(t,y) = t^2 + ty \).
    Esta ecuación diferencial no es exacta desde que 
    \( \frac{\partial M}{\partial y} = 3t + 2 y \) y 
    \( \frac{\partial N}{\partial t} = 2t +y \).
    Pero aún podemos encontrar un factor integrante.
    Supongamos que existe \( \mu \), función de \( t \) y de \( y \) tal que
    \[
        \frac{\partial}{\partial y } 
        \left[ 
            \mu(t,y)
            M(t,y)
        \right] 
        =
        \frac{\partial}{\partial t} 
        \left[ 
            \mu(t,y)N(t,y)
        \right],
    \]
    o bien 
    \[
        M
        \frac{\partial \mu}{\partial y} +
        \mu
        \frac{\partial M}{\partial y} 
        =
        N
        \frac{\partial \mu}{\partial t} 
        +
        \mu
        \frac{\partial N}{\partial t}.
    \]
    Si además $\mu$ es sólo función de $t$, su derivada parcial con respecto a
    \( y \) se anula, y nos deja con
    \[
        \mu
        \frac{\partial M}{\partial y} 
        =
        N
        \frac{\partial \mu}{\partial t} 
        +
        \mu
        \frac{\partial N}{\partial t}.
    \]
    Esto implica que 
    \[
        N \frac{\partial \mu}{\partial t} 
        =
        \mu 
        \left( 
            \frac{\partial M}{\partial y} 
            -
            \frac{\partial N}{\partial t} 
        \right).
    \]
    Si $\mu\neq 0$ para todo su dominio,  podemos ver que 
    \begin{align*}
        \frac{1}{\mu} \frac{\partial \mu}{\partial t} 
        &=
        \frac{
            \frac{\partial M}{\partial y} 
            -
            \frac{\partial N}{\partial t} 
        }{N}\\
        &=
        \frac{
            3t+2y-2t-y
        }{t^2+ty} \\
        &=
        \frac{t+y}{t(t+y)} \\
        &=
        \frac{1}{t}.
    \end{align*}
    Esta ecuación es más sencilla, pues al integrar con respecto a $t$ obtenemos que 
    \(
        \mu(t)
        =
        t.
    \)
    Entonces tenemos que 
    \begin{align*}
        0
        &=
        \mu(t)
        \left( 
        3ty+y^2
        +
        ( t^2 + ty ) 
        \frac{dy}{dt}
        \right) \\
        &=
        3t^2y+ty^2 
        +
        \left( 
            t^3 + t^2 y
        \right) \frac{\partial y}{\partial t}
    \end{align*}
    es una ecuación diferencial exacta e identificamos 
    \( M^*(t,y) = 3t^2 y + ty^2 \) y a \( N^* ( t, y ) = t^3 +t^2 y \).
    Luego, integrando con respecto a $t$ tenemos
    \begin{align*}
        \Phi(t,y)
        &=
        \int M^*(t,y) \, dt
        +
        h(y),\quad\text{para alguna función $h$,}\\
        &=
        \int
        3t^2y+ty^2\,dt
        +h(y)\\
        &=
        t^3 y + \frac{t^2 y^2}{2} + h(y).
    \end{align*}
    Derivando con respecto a $y$ obtenemos que 
    \[
        h'(y)
        +t^3 + t^2y = \frac{\partial \Phi}{\partial y} = N^*(t,y)=t^3+t^2 y.
    \]
    Por lo tanto $h'(y)=0$, y $h$ es constante. Como \( \Phi(t,y) = c\in\R \), tenemos que 
    la solución general de la ecuación diferencial es 
    \[
        c' = t^3 y + \frac{t^2 y^2}{2}, \quad\text{con $c'\in\R$}.
    \]
    Sin embargo, la condición inicial nos dice que 
    \begin{align*}
        c'
        &=
        2^3 \cdot 1
        +
        2 \cdot 1= 10.
    \end{align*}
    Por lo tanto, la solución de esta ecuación diferencial es 
    \[
        10 = t^3 y(t) +
        \frac{t^2 y^2(t)}{2}.
    \]
\end{solution}
