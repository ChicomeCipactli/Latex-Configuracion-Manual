Suponga que \( A \) es el set de enteros divisibles por \( 4 \). 
Similarmetne suponga que \( B \) y \( C \) son los conjuntos de enteros divisibles
por \( 9 \) y \( 10 \), respectivamente. ¿Qué es \( A \cap B \cap C \)?
\begin{solution}
    Sea \( D = \set{ n \in \Z: n \text{ es múltiplo de }180 } \). Este es el conjunto 
    de la intersección, es decir
    \[
        D = A \cap B \cap C
    \]
    Sea \( n \in D \). Como \( n \) es divisible por 180, se puede escribir como
    \begin{eqnarray*}
        n 
        &=&
        180 k,\quad\text{para algún entero }k\\
        &=&
        ( 9 ) ( 2 )^2 ( 5 ) k
    \end{eqnarray*}
    Entonces \( 9,\) divide a \( n \), por lo que \( n\in B \), 
    \( 2^2 = 4 \) divide a \( n \), por lo que \( n \in A \), y por último 
    \( 2 \cdot 5 = 10 \) divide a \( n \), por lo cual \( n \in C \).
    Así que \( n \in A\cap B \cap C \). 
    Esto prueba que \( D \subseteq A \cap B \cap C \).

    Por otro lado si \( n \in A \cap B \cap C \), entonces
    \( n \in A \), con lo que \( n \) es divisible por 4, y eso implica que 
    se puede escribir como \( n = 4 \cdot k' \) para algún \( k'\in \Z \).
    También puede concluirse que \( n \in B \), con lo cual \( n \) es divisible
    por \( 9 \), y entonces \( 9 \) debe dividir a 
    \( k' \), ya que \( 9 \) no tiene divisores primos en común con \( 4 \). Por lo tanto 
    \( n = 9 \cdot 4 \cdot k''\), con \( k'' \) entero. Por último, como también
    \( n \) es divisible por 10. Sin embargo, como \( 2 \) divide a \( 4 \) no 
    es necesario que \( 10 \) divida a \( k'' \). Sin embargo, \( 5 \) no tiene factores en común
    con \( 9 \cdot 4 \) por lo que \( 5 \) divide a \( k'' \). Entonces, 
    \[
        n = 9 \cdot 4 \cdot 5 \cdot k'''',
    \]
    para algún entero \( k'''' \). Entonces \( 9 \cdot 5 \cdot 4 = 180\) 
    divide a \( n \). Por tanto \( n\in D \). Concluimos que 
    \( A \cap B \cap C \subseteq D\), y por tener doble contención 
    \[
        D = A \cap B \cap C.
    \]
\end{solution}
