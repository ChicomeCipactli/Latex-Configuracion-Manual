\setcounter{section}{1}
\begin{enumerate}[(i)] \item
        Sea \( \alpha = \left( i_0\quad i_1\quad \cdots\quad i_{r-1} \right) \) un \( r \)-ciclo.
        Para cada \( j,k\geq 0 \), demuestra que \( \alpha^k \left( i_j \right) = i_{k+j} \) 
        si los subíndices se leen módulo \( r \).
        \begin{proof}
            Podemos identificar al conjunto de
            \( \set{0,\ldots,r-1} \) con \( \Z_{r} \) para así decir que 
            \( i_{j}= i_{\overline{t}} \) siempre que \( j\in \overline{t} \), 
            donde \( j\in\set{1,\ldots,r-1} \) y \( \overline{t}\in\Z_{r} \), es
            decir, \( j \equiv t \mod r  \). 
            % Por lo tanto, es cierto el resultado
            % para \( k = 0 \).

            Procederemos por inducción sobre \( k\geq 0 \). Para \( k = 0 \) 
            tenemos que \( \alpha^k = \mathrm{id} \). Por lo tanto, para toda $j\geq 0$
            se tiene \( \alpha^k \left( i_{\overline{j}} \right) = i_{\overline{j}}=
            i_{\overline{j+k}}\). Así que nuestra base inductiva con \( k=0 \) es cierta.
            
            Supongamos que para alguna \( k\geq 0 \) se tiene que para toda \( \overline{j}\in
            \Z_{r}\), \( \alpha^k \left( i_{\overline{j}} \right) = i_{\overline{j+k}} \).
            Al aplicar nuevamente \( \alpha \) obtenemos que para toda \( \overline{t}\in
            \Z_{r}\) 
            \begin{align*}
                \alpha^{k+1}
                \left( i_{\overline{t}} \right) 
                &=
                \alpha 
                \left( 
                    \alpha^k
                    \left(
                    i_{\overline{t}}
                    \right)
                \right) \\
                &=
                \alpha \left( i_{\overline{t+k}} \right) ,\quad\text{por la hipótesis de inducción,}\\
            \end{align*}
            Sin embargo, como \( \alpha \) es un ciclo, 
            tenemos \( \alpha \left( i_{\overline{t+k}} \right) 
            = i_{\overline{t+k}+\overline{1}}
            = i_{\overline{t+k+1}}\).
            Por lo tanto 
            \( \alpha^{k+1} \left( i_{\overline{j}} \right) 
            = i_{\overline{j+k+1}}\). Por el principio de inducción, se satisface lo anterior 
            para cada \( k\geq 0 \).
        \end{proof}
    \item
        Demuestra que si \( \alpha \) es un \( r \)-ciclo,
        entonces \( \alpha^r=1 \), pero que \( \alpha^k \neq 1 \) para cada 
        entero positivo \( k<r \).
        \begin{proof}
            Sea \( i_{\overline{j}}\) un elemento que se mueve por el ciclo \( \alpha \).
            Por el inciso anterior
            \[
                \alpha^r \left( i_{\overline{j}} \right) 
                =
                i_{\overline{j+r}}
                =
                i_{\overline{j}+\overline{r}}
                =
                i_{\overline{j}+\overline{0}}
                =
                i_{\overline{j}}.
            \]
            Para cualquier \( i_{\overline{j}} \) que se queda fijo por \( \alpha \), 
            para toda \( k\geq 0 \) también se queda fijo por \( \alpha^k \),
            en particular para \( k = r \).
            De esta manera \( \alpha^r \) es la función identidad porque fija a todos los elementos
            en su dominio.
            
            Sin embargo, si \( 0<k<r \), por lo anterior \( \alpha^k \left( i_{\overline{0}} \right)
            = i_{\overline{k}}\), pero \( \overline{k} \neq \overline{0} \), y
            entonces \( \alpha^k \left( i_{\overline{0}} \right) \neq i_{\overline{0}} \). 
            Así que \( \alpha^k\) no es la identidad.
        \end{proof}
    \item
        Si \( \alpha=\beta_1\beta_2\cdots \beta_m \) es un producto de \( r_i \)-ciclos 
        disuntos \( \beta_i \), entonces el 
        entero positivo más pequeño \( l \) con \( \alpha^l = 1\) es el mínimo común múltiplo de 
        \( \set{r_1,r_2,\ldots,r_m} \).

        % Usaremos el siguiente lema.
        % \begin{lemma}
        %     Si \( \beta_1,\ldots,\beta_m \) son permutaciones disjuntas por pares,
        %     entonces \( \beta_m \) y \( \beta_1\cdots\beta_{m-1} \) son disjuntas.
        % \end{lemma}
        % \begin{proof}
        %     % Por inducción sobre \( m\geq 2 \). Por hipótesis, el caso de \( m = 2 \) es cierto.
        %     % Supongamos que para alguna \( k \geq 2 \),
        %     % si se tienen
        %     % \( \beta_1,\ldots, \beta_k \) ciclos disjuntos por pares, entonces 
        %     % \( \beta_k \) y \( \beta_{1}\cdots \beta_{k-1} \) son disjuntos.
        %     % Si \( \beta_{k+1} \) es otro ciclo que es disjunto a cada \( \beta_i \) para toda
        %     % \( i(1 \leq i \leq k) \) {}
        %     Sea \( i \) tal que \( \beta_m ( i ) \neq i \). Entonces
        %     se tiene \( i = \beta_1 ( i ) = \cdots = \beta_{m-1} ( i ) \).
        %     Así que 
        %     \begin{align*}
        %         \left(
        %         \prod_{j=1}^{m-1} \beta_ j \right)(i)
        %         &=
        %         \left( \prod_{j=2}^{m-1} \beta_j \right) \left( \beta_1 ( i ) \right) \\
        %         &=
        %         \left( \prod_{j=2}^{m-1} \beta_j \right) \left( i  \right) \\
        %         &\quad\vdots \\
        %         &=
        %         \beta_{m-1} ( i ) \\
        %         &= i.
        %     \end{align*}
        %     Por lo tanto \( \beta_{m-1}\cdots \beta_{1} \) fijan a \( i \). Entonces
        %     \( \beta_{m} \) y \( \beta_{m-1}\cdots \beta_{1} \) son disjuntas.
        % \end{proof}

        % El siguiente lema también será de utilidad.
        % \begin{lemma}
        %     Sean \( \beta_1,\ldots,\beta_m \) permutaciones disjuntas por pares.
        %     Entonces 
        %     \[
        %         \left( \prod_{i=1}^{m} \beta_i \right)^n
        %         =
        %         \prod_{i=1}^{m}
        %         \beta_i^{n},\quad\text{para toda $n\in \Z$.}
        %     \]
        % \end{lemma}
        % \begin{proof}
        %     El caso \( n = 0 \) es trivial, por componer \( m \) veces la función identidad.
        % \end{proof}
        \begin{proof}
            Por inducción se puede probar que \( \alpha^n = \beta_1^n \cdots \beta_m^n \) para
            toda \( n\in \Z \), como una generalización del ejercicio en la tarea 1,
            pues es el producto de permutaciones disjuntas.

            De esta manera tenemos \( 1=\alpha^l = \beta_1^l \cdots \beta_m^l \). 
            Si algún \( \beta_i \) mueve a \( x \), entonces \( \beta_j \) lo fija 
            si \( i \neq j \). Por lo tanto \( \beta_{j}^l \) también lo deja fijo, 
            ya que \( \beta^l( x ) = \beta^{l-1} \left( \beta(x) \right) = \beta^{l-1} 
            \left( x \right) \), e inductivamente se tiene que \( \beta^{l-1} ( x ) = x \).

            Informalmente... \( \beta_i^l (x) = 1 \). Entonces \( l \) es un múltiplo de 
            \( r_i \) por incisos anteriores, 
            pues esto se mantiene para toda \( x \) que sea movida por \( \beta_i \).
            Por tanto \( l \) es un común múltiplo de \( r_1,\ldots,r_m \).
            
            Si el mínimo común múltiplo lo satisface entonces terminamos. Pero sí lo hace
            porque se elevan cada una de las betas al mcm de las \( r_i \)'s. Lo cual
            las hace identidades.
        \end{proof}
\end{enumerate}
 
