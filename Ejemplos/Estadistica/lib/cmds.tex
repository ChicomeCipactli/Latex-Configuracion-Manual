% Conjuntos

\newcommand{\R}{
    \mathbb{R}
}
\newcommand{\N}{
    \mathbb{N}
}
\newcommand{\Z}{
    \mathbb{Z}
}
\newcommand{\Q}{
    \mathbb{Q}
}
\newcommand{\F}{
    \mathbb{F}
}
\newcommand{\C}{
    \mathbb{C}
}
\newcommand{\Fs}{
    \mathcal{F}
}
    % Funciones
    \newcommand{\imagen}[1]{
        \mathrm{Im}
        \left(
            #1
        \right)
    }
    \newcommand{\dominio}[1]{
        \mathrm{Dom}
        \left(
            #1
        \right)
    }

    % Conjuntos de Álgebra Lineal
    \newcommand{\kernel}[1]{
        \mathrm{Ker}
        \left(
            #1
        \right)
    }
    \newcommand{\setspan}[1]{
        \mathrm{span}
        \left(
            #1
        \right)
    }
    \newcommand{\ltspace}[1]{
        \mathcal{L}
        \left(
            #1
        \right)
    }


% \newcommand{\Borel}[1]{
%     \mathcal{B}\left(#1\right)
% }
\newcommand{\set}[1]{
    \left\{
        #1
    \right\}
}

% Funciones
    % Normas
    \providecommand{\abs}[1]{
        \left\vert
            #1
        \right\vert
    }
    \providecommand{\norm}[1]{
        \left\lVert
            #1
        \right\rVert
    }
    % \providecommand{\abs}[1]{
    %     \left\lvert
    %         #1
    %     \right\rvert
    % }
    \providecommand{\prodint}[1]{
        \left\langle
            #1
        \right\rangle
    }
    % Probabilidad
    % \newcommand{\Esp}[1]{
    %     \mathbb{E}
    %     \left[
    %         #1
    %     \right]
    % }
    \newcommand{\Prob}[1]{
        \mathbb{P}
        \left[
            #1
        \right]
    }
    \newcommand{\ind}[2]{
        \mathbb{I}_{#1}
        \left(
            #2
        \right)
    }
    % Integrales
    \newcommand{\riemann}[4]{
        \int
        % límites
        \limits_{#1}^{#2}
        % integrando
        #3
        % referencia
        \,d#4
    }
    \newcommand{\lebesgue}[3]{
        \int
        % límites ó conjunto
        \limits_{#1}
        % integrando
        #2
        % referencia
        \,d#3
    }
    \newcommand{\stieltjes}[3]{
        \int
        % limites
        \limits_{#1}^{#2}
        % integrando
        #3
        % referencia
        \,F(dx)
    }
    % Mapeos
    \newcommand{\To}{
        \longrightarrow
    }%-->
    \newcommand{\mTo}{
        \longmapsto
    } %I-->


% Lógica
    \newcommand{\ssi}{
        \Leftrightarrow
    }
    \newcommand{\ent}{
        \Rightarrow
    } %=>
    \newcommand{\Ent}{
        \Longrightarrow
    }%==>
    \newcommand{\EntL}{
        \Longleftarrow
    }%<==

% Letras especiales
\newcommand{\ohm}{
    \omega
}
\newcommand{\Ohm}{
    \Omega
}
\newcommand{\eps}{
    \varepsilon
}

% Álgebra Lineal
    % Dimensiones
    \newcommand{\rango}[1]{
        \mathrm{rango}
        \left(
            #1
        \right)
    }
    \newcommand{\nulidad}[1]{
        \mathrm{nulidad}
        \left(
            #1
        \right)
    }
    \renewcommand{\dim}[2][]{
        \mathrm{dim}_{#1}
        \left(
            #2
        \right)
    }


